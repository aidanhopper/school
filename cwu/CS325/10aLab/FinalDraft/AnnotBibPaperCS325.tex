\documentclass[12pt,letterpaper]{article}

% just for the example
\usepackage{lipsum}
% Set margins to 1.5in
\usepackage[margin=1.5in]{geometry}

% for graphics
\usepackage{graphicx}

% for crimson text
\usepackage{crimson}
\usepackage[T1]{fontenc}

% setup parameter indentation
\setlength{\parindent}{0pt}
\setlength{\parskip}{6pt}

% for 1.15 spacing between text
\renewcommand{\baselinestretch}{1.15}

% For defining spacing between headers
\usepackage{titlesec}
% Level 1
\titleformat{\section}
  {\normalfont\fontsize{18}{0}\bfseries}{\thesection}{1em}{}
% Level 2
\titleformat{\subsection}
  {\normalfont\fontsize{14}{0}\bfseries}{\thesection}{1em}{}
% Level 3
\titleformat{\subsubsection}
  {\normalfont\fontsize{12}{0}\bfseries}{\thesection}{1em}{}
% Level 4
\titleformat{\paragraph}
  {\normalfont\fontsize{12}{0}\bfseries\itshape}{\theparagraph}{1em}{}
% Level 5
\titleformat{\subparagraph}
  {\normalfont\fontsize{12}{0}\itshape}{\theparagraph}{1em}{}
% Level 6
\makeatletter
\newcounter{subsubparagraph}[subparagraph]
\renewcommand\thesubsubparagraph{%
  \thesubparagraph.\@arabic\c@subsubparagraph}
\newcommand\subsubparagraph{%
  \@startsection{subsubparagraph}    % counter
    {6}                              % level
    {\parindent}                     % indent
    {12pt} % beforeskip
    {6pt}                           % afterskip
    {\normalfont\fontsize{12}{0}}}
\newcommand\l@subsubparagraph{\@dottedtocline{6}{10em}{5em}}
\newcommand{\subsubparagraphmark}[1]{}
\makeatother
\titlespacing*{\section}{0pt}{12pt}{6pt}
\titlespacing*{\subsection}{0pt}{12pt}{6pt}
\titlespacing*{\subsubsection}{0pt}{12pt}{6pt}
\titlespacing*{\paragraph}{0pt}{12pt}{6pt}
\titlespacing*{\subparagraph}{0pt}{12pt}{6pt}
\titlespacing*{\subsubparagraph}{0pt}{12pt}{6pt}

% Set caption to correct size and location
\usepackage[tableposition=top, figureposition=bottom, font=footnotesize, labelfont=bf]{caption}

% set page number location
\usepackage{fancyhdr}
\fancyhf{} % clear all header and footers
\renewcommand{\headrulewidth}{0pt} % remove the header rule
\rhead{\thepage}
\pagestyle{fancy}

% Overwrite Title
\makeatletter
\renewcommand{\maketitle}{\bgroup
   \begin{center}
   \textbf{{\fontsize{18pt}{20}\selectfont \@title}}\\
   \vspace{10pt}
   {\fontsize{12pt}{0}\selectfont \@author} 
   \end{center}
}
\makeatother

% Used for Tables and Figures
\usepackage{float}

% For using lists
\usepackage{enumitem}

% For full citations inline
\usepackage{bibentry}
\nobibliography*

% Custom Quote
\newenvironment{myquote}[1]%
  {\list{}{\leftmargin=#1\rightmargin=#1}\item[]}%
  {\endlist}
  
% Create Abstract 
\renewenvironment{abstract}
{\vspace*{-.5in}\fontsize{12pt}{12}\begin{myquote}{.5in}
\noindent \par{\bfseries \abstractname.}}
{\medskip\noindent
\end{myquote}
}



% Set Title, Author, and email
\title{Annotated Bibliography; CS325}
\author{A Hopper \\ hoppera@cwu.edu}
\date{\today}

\begin{document}

\maketitle
\thispagestyle{fancy}

\section*{Topic Proposal: }
My proposal is "How Open Source Software Has Changed Computer Science". Open source software is defined as software that anyone can inspect, modify, and change. The reason open source is so impactful is because of its public-good nature. Open source software is available to be freely used by anyone and contributed to by anyone. 

% There are four placeholder entries below for your Annotated Bibliography. Please remember that you will need at least five for your rough draft and ten for your final version.

\subsection*{\bibentry{CAULKINS20131182}}
This article explores when proprietary software should be made open source. The article found that open sourcing proprietary software can have substantial on quality. The article shows that high R&D costs for proprietary software will usually cause the software to be open sourced. When R&D is inexpensive then firms will typically only open source their software when the quality of software is low.


\subsection*{\bibentry{DALLE20031}}
This article analyzes the adoption of "Libre" software. Libre is defined as free and open source with the right to anyone to modify and redistribute. The article finds that if Libre alternatives to proprietary software are considered it would improve social welfare. The article suggests the use of public intervention to foster the use of Libre solutions.

\subsection*{\bibentry{12429789120170701}}
This article analyzed the adoption of open source versus proprietary software. The article says that users have different reasons for adopting software that follows the two development models. For example, users who prefer more security will gravitate towards open source solutions, while users who prefer stability and commercial compatibility will gravitate towards proprietary solutions. 

\subsection*{\bibentry{2151758820060701}}
This article explores pricing strategies for open source software and proprietary software. The article compared platforms such as Windows and Linux to find compare the two development models. It was found that proprietary platforms are more likely to dominate markets.

\subsection*{\bibentry{KILAMO20121467}}
This article is meant to provide insight into establishing and maintaining open source communities. One of the key findings on this article is how open source communities operate. Open source communities have an onion model for development, with each layer specifying their influence on the project. On the outside of this onion there is the user who does not interact with the code at all. The next layer is the reader who will read the code, but will not contribute. This goes all the way down to project leader, the most influential figure to the project.

\subsection*{\bibentry{ISRAELI2010485}}
This article describes the evolution of the Linux kernel, the largest open source software project. Linux is a special type of software that has embedded itself into the infrastructure of everyday life, making it an E-type system, meaning it will be in perpetual development. The growth of the Linux kernel was analyzed and it was found that the growth was super-linear, meaning that the kernel growth rate is very fast. The article uses Lehman's law of software evolution to understand the kernel's growth rate.

\subsection*{\bibentry{LAKHANI2003923}}
There are many mundane but necessary tasks in open-source development and the article explores the motivations contributors have to perform them. One of the biggest reasons to perform these mundane tasks is because the contributors are also users of the software. Some of the other reasons cited are that the contributors enjoy helping others, enjoy solving problems, and enjoy the respect earned by performing said tasks.

\subsection*{\bibentry{OSTERLOH2007157}}
This article explores open-source as an innovation model, that differs a lot from other typical innovation model. A typical innovation model would be motivated by price systems, while open-source is not. The article summarizes open-source as collective innovation, which usually did not survive past a dominant design, but open-source software seems to circumvent this issue.

\subsection*{\bibentry{11179983820151201}}
Michael Sacks is an assistant professor at the University of Louisville with a Ph. D. in Economics. Sacks' article explains how open source software effects its proprietary competition. Proprietary software is developed behind closed doors, instead of everyone being able to view and contribute to the code, only a selected few can. The article found that proprietary software competing with open source software is very different that proprietary competing with proprietary because the open source community is not a profit maximizing entity.

\subsection*{\bibentry{kroghhippel}}
This article explains the incentives behind contributing to open source software. One of the key findings of the article is that proprietary software based on an open source platform is more profitable than a platform that is proprietary. This means that even proprietary firms like Microsoft have an incentive to contribute to open source software included in their products because it increase profits. The article also describes how open source software is a public good an non rival. This means that the practice does societal good.

\bibliographystyle{apalike2} 
\nobibliography{bibtemp}


\end{document}
